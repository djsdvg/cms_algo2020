% Template per generare 

\documentclass[a4paper,11pt]{article}
\usepackage{lmodern}
\renewcommand*\familydefault{\sfdefault}
\usepackage[utf8]{inputenc}
\usepackage[T1]{fontenc}
\usepackage[italian]{babel}
\usepackage{indentfirst}
\usepackage{graphicx}
\usepackage{tikz}
\newcommand*\circled[1]{\tikz[baseline=(char.base)]{
		\node[shape=circle,draw,inner sep=2pt] (char) {#1};}}
% \usepackage[group-separator={\,}]{siunitx}
\usepackage[left=2cm, right=2cm, bottom=3cm]{geometry}
\frenchspacing

\newcommand{\num}[1]{#1}

% Macro varie...
\newcommand{\file}[1]{\texttt{#1}}
\renewcommand{\arraystretch}{1.3}
\newcommand{\esempio}[2]{
\noindent\begin{minipage}{\textwidth}
\begin{tabular}{|p{11cm}|p{5cm}|}
	\hline
	\textbf{File \file{input.txt}} & \textbf{File \file{output.txt}}\\
	\hline
	\tt \small #1 &
	\tt \small #2 \\
	\hline
\end{tabular}
\end{minipage}
}

% Dati del task
\newcommand{\nome}{Mappa antica}
\newcommand{\nomebreve}{mappa}

\begin{document}
% Intestazione
\noindent{\Huge \textbf \nome~(\texttt{\nomebreve})}\\
{\large \textbf (preso ed adattato dalla Selezione Territoriale 2008 delle OII)}


% Descrizione del task
\section*{Descrizione del problema}
Topolino è in missione per accompagnare una spedizione archeologica che segue un'antica mappa
acquisita di recente dal museo di Topoinia. Raggiunta la località dove dovrebbe trovarsi un
prezioso e raro reperto archeologico, Topolino si imbatte in un labirinto che ha la forma di una
gigantesca scacchiera quadrata di NxN lastroni di marmo.

Nella mappa, sia le righe che le colonne del labirinto sono numerate da 1 a N. Il lastrone che si
trova nella posizione corrispondente alla riga r e alla colonna c viene identificato mediante la
coppia di interi $(r, c)$. I lastroni segnalati da una crocetta '+' sulla mappa contengono un
trabocchetto mortale e sono quindi da evitare, mentre i rimanenti sono innocui e segnalati da un
asterisco '*'.

Topolino deve partire dal lastrone in posizione (1, 1) e raggiungere il lastrone in posizione (N, N),
entrambi innocui. Può passare da un lastrone a un altro soltanto se questi condividono un lato o
uno spigolo (quindi può procedere in direzione orizzontale, verticale o diagonale ma non saltare) e,
ovviamente, questi lastroni devono essere innocui.

Tuttavia, le insidie non sono finite qui: per poter attraversare incolume il labirinto, Topolino deve
calpestare il minor numero possibile di lastroni innocui (e ovviamente nessun lastrone con
trabocchetto). Aiutate Topolino a calcolare tale numero minimo.

% Input
\section*{File di input}
Il programma deve leggere da un file di nome \file{input.txt}. La prima riga contiene un intero positivo che rappresenta la dimensione N di un lato del labirinto a scacchiera. 
Le successive N righe rappresentano il labirinto a scacchiera: la $r$-esima di tali righe contiene una
sequenza di N caratteri '+' oppure '*', dove '+' indica un lastrone con trabocchetto mentre '*' indica
un lastrone sicuro. Tale riga rappresenta quindi i lastroni che si trovano sulla $r$-esima riga della
scacchiera: di conseguenza, il $c$-esimo carattere corrisponde al lastrone in posizione $(r, c)$. I caratteri NON sono separati da degli spazi.

% Output
\section*{File di output}
Il programma deve scrivere in un file di nome \file{output.txt}. Deve venire stampato il minimo numero di lastroni innocui (ossia indicati con '*') che Topolino deve attraversare a partire dal
lastrone in posizione (1, 1) per arrivare incolume al lastrone in posizione (N, N). Notare che i
lastroni (1, 1) e (N, N) vanno inclusi nel conteggio dei lastroni attraversati.

% Assunzioni
\section*{Assunzioni}
\begin{itemize}
\item $1 \leq N \leq 100$
\item $1 \leq r,c \leq N$
\end{itemize}

% Subtasks
\section*{Subtask}
\begin{itemize}
\item \textbf{Subtask 1 [\phantom{1}0 punti]:} caso di esempio.
\item \textbf{Subtask 2 [20 punti]:} $N \le 10$.
\item \textbf{Subtask 3 [30 punti]:} $N \le 20$.
\item \textbf{Subtask 4 [20 punti]:} $N \le 50$.
\item \textbf{Subtask 5 [30 punti]:} $N \le 100$.
\end{itemize}


% Esempi
\section*{Esempio di input/output}
\esempio{
5

***+*

+**++

*+*+*

+++*+

+**+*
}{5}
\end{document}

\usepackage{xcolor}
\usepackage{afterpage}
\usepackage{pifont,mdframed}
\usepackage[bottom]{footmisc}


\createsection{\Grader}{Grader di prova}

\newcommand{\inputfile}{\texttt{input.txt}}
\newcommand{\outputfile}{\texttt{output.txt}}

\newenvironment{warning}
  {\par\begin{mdframed}[linewidth=2pt,linecolor=gray]%
    \begin{list}{}{\leftmargin=1cm
                   \labelwidth=\leftmargin}\item[\Large\ding{43}]}
  {\end{list}\end{mdframed}\par}

Calcolare il numero di componenti fortemente connesse di un dato un grafo orientato.

\Implementation


Dovrai sottoporre esattamente un file con estensione \texttt{.c} o \texttt{.cpp}.

\begin{warning}
Tra gli allegati a questo task troverai un template (\texttt{scc.c}, \texttt{scc.cpp}) con un esempio di implementazione.
\end{warning}

Dovrai implementare la seguente funzione:

\begin{center}\begin{tabularx}{\textwidth}{|c|X|}
\hline
C/C++  & \verb|int scc(int N, int E, int da[], int a[]);|\\
\hline
\end{tabularx}\end{center}

\begin{itemize}[nolistsep]
  \item L'intero $N$ rappresenta il numero di vertici del grafo.
  \item L'intero $E$ rappresenta il numero di archi del grafo.
  \item Gli array \texttt{da} e \texttt{a}, indicizzati da $0$ a $E-1$, contengono gli estremi di ogni arco del grafo, cioè per ogni $i$ fra $0$ ed $E-1$ esiste un arco che parte da \texttt{da}$[i]$ e arriva a \texttt{a}$[i]$. I vertici sono indicizzati da $0$ a $N-1$
  \item La funzione dovrà restituire il numero di componenti fortemente connesse.
\end{itemize}

\medskip

Il grader chiamerà prima la funzione \texttt{scc} e ne stamperà il valore restituito sul file di output.

% % % % % % % % % % % % % % % % % % % % % % % % % % % % % % % % % % % % % % % % % % %
% % % % % % % % % % % % % % % % % % % % % % % % % % % % % % % % % % % % % % % % % % %


\Grader
Nella directory relativa a questo problema è presente una versione semplificata del grader usato durante la correzione, che potete usare per testare le vostre soluzioni in locale. Il grader di esempio legge i dati di input dal file \texttt{input.txt}, chiama le funzioni che dovete implementare e scrive il file \outputfile{}, secondo il seguente formato.

Il file \inputfile{} è composto da $E+1$ righe, contenenti:
\begin{itemize}[nolistsep,itemsep=2mm]
\item Riga $1$: gli interi $N$ ed $E$ separati da uno spazzio.
\item Righe da $2$ a $E+1$: contengono ognuna due interi che indicano che esiste un grafo con vertici (orientati) i due numeri.
\end{itemize}

Il file \outputfile{} è composto da un'unica riga, contenente:
\begin{itemize}[nolistsep,itemsep=2mm]
\item Riga $1$: il valore restituito dalla funzione \texttt{scc}.
\end{itemize}

% % % % % % % % % % % % % % % % % % % % % % % % % % % % % % % % % % % % % % % % % % %
% % % % % % % % % % % % % % % % % % % % % % % % % % % % % % % % % % % % % % % % % % %


\Constraints

\begin{itemize}[nolistsep, itemsep=2mm]
	\item $1 \le N \le 10\,000$.
	\item $1 \le E \le 100\,000$.
\end{itemize}

% % % % % % % % % % % % % % % % % % % % % % % % % % % % % % % % % % % % % % % % % % %
% % % % % % % % % % % % % % % % % % % % % % % % % % % % % % % % % % % % % % % % % % %


\Examples

\begin{example}
\exmp{
8 8
0 4
4 5
5 6
6 7
7 2
2 3
3 4
1 2
}{%
3
}%
\end{example}

% % % % % % % % % % % % % % % % % % % % % % % % % % % % % % % % % % % % % % % % % % %
% % % % % % % % % % % % % % % % % % % % % % % % % % % % % % % % % % % % % % % % % % %

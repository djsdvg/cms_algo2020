% Template per generare

\documentclass[a4paper,11pt]{article}
\usepackage{lmodern}
\renewcommand*\familydefault{\sfdefault}
\usepackage{sfmath}
\usepackage[utf8]{inputenc}
\usepackage[T1]{fontenc}
\usepackage[italian]{babel}
\usepackage{indentfirst}
\usepackage{graphicx}
\usepackage{tikz}
\newcommand*\circled[1]{\tikz[baseline=(char.base)]{
    \node[shape=circle,draw,inner sep=2pt] (char) {#1};}}
\usepackage{enumitem}
% \usepackage[group-separator={\,}]{siunitx}
\usepackage[left=2cm, right=2cm, bottom=3cm]{geometry}
\frenchspacing

\newcommand{\num}[1]{#1}

% Macro varie...
\newcommand{\file}[1]{\texttt{#1}}
\renewcommand{\arraystretch}{1.3}
\newcommand{\esempio}[2]{
  \noindent\begin{minipage}{\textwidth}
    \begin{tabular}{|p{11cm}|p{5cm}|}
      \hline
      \textbf{File \file{input.txt}} & \textbf{File \file{output.txt}}\\
      \hline
      \tt \small #1 &
      \tt \small #2 \\
      \hline
    \end{tabular}
  \end{minipage}
}

% Dati del task
\newcommand{\gara}{.}
\newcommand{\nome}{Individuare il Max in Versione Interattiva}
\newcommand{\nomebreve}{max\_x}

\begin{document}
  
  
  % Intestazione
  \noindent{\Large \gara}
  \vspace{0.5cm}
  
  \noindent{\Huge \textbf \nome~(\texttt{\nomebreve})}
  \vspace{0.2cm}\\
  
  % Descrizione del task
  \section*{Descrizione del problema}
    
  \noindent
  Devi scrivere una procedura che individui la biglia di peso massimo
  in un set di $n$ biglie numerate da $0$ ad $n-1$.
  Il file da sottomettere deve avere la seguente struttura:
\begin{verbatim}
#include "ourLibToPlay.h"

void individua(long int n, long int maxLies) {
   ...
} 
\end{verbatim}

  \noindent
  Il parametro $n$ che viene passato alla funzione \texttt{individua} da te realizzata \`e il numero di biglie sotto esame. Le $n$ biglie differiscono tutte per peso.
  Potrai servirti di una bilancia a braccia eguali
  invocando, dalla tua implementazione della procedura \texttt{individua},
  la seguente funzione:
  
  \vspace{0.2cm}
  
  \noindent
  \texttt{int pesa(long int bigliaA, long int bigliaB)}
  
  \vspace{0.2cm}
  
  \noindent
  La funzione \texttt{pesa} ritorna
  \begin{itemize}[noitemsep]
    \item $-1$, se \texttt{bigliaA} \`e pi\`u leggera di \texttt{bigliaB};
    \item $1$,  se \texttt{bigliaA} \`e pi\`u pesante di \texttt{bigliaB}.
  \end{itemize}

  Nei subtask pi\`u impegnativi, la bilancia pu\`o
  talvolta mentire su quale delle due biglie sia effettivamente la pi\`u pesante. \`E tuttavia sempre garantito che il numero di tali bugie o malfunzionamenti non ecceder\`a mai il valore di \texttt{maxLies}, il secondo parametro in input della tua funzione \texttt{individua}.   
  
  \noindent
  Una volta individuata senza alcun margine di dubbio la biglia di peso massimo,
  devi consegnarla invocando la funzione:
  
  \vspace{0.2cm}
  
  \noindent
  \texttt{void pensoCheMaxSia(long int bigliaMax)}
  
  
  \section*{Subtask}
  \begin{itemize}
    \item \textbf{Subtask 0 [1 punti]:} la biglia pi\`u pesante è la $2$.
    \item \textbf{Subtask 1 [2 punti]:} $n=2$, bilancia infallibile ed onesta (\texttt{maxLies}$= 0$).
    \item \textbf{Subtask 2 [4 punti]:} \texttt{maxLies}$= 0$, sono consentite al pi\`u $n$ pesate.
    \item \textbf{Subtask 3 [8 punti]:} \texttt{maxLies}$= 0$, sono consentite al pi\`u $n-1$ pesate.
    \item \textbf{Subtask 4 [16 punti]:} la bilancia pu\`o mentire, ma al pi\`u una sola volta (\texttt{maxLies}$\leq 1$).
    \item \textbf{Subtask 5 [5 punti]:} tolleranza ad al pi\`u una bugia (\texttt{maxLies}$\leq 1$), e in al pi\`u $3n-3$ pesate.
    \item \textbf{Subtask 6 [32 punti]:} \texttt{maxLies}$\leq 1$, e in al pi\`u $2n$ pesate.
    \item \textbf{Subtask 7 [16 punti]:} \texttt{maxLies}$\leq 1$, e in al pi\`u $2n-1$ pesate.
    \item \textbf{Subtask 8 [16 punti]:} tolleranza ad al pi\`u $k$ bugie (\texttt{maxLies}$= k$ generico), e in al pi\`u $(k+1)(n-1) +k = n(k+1)-1$ pesate.
  \end{itemize}
  
  % Assunzioni
  \section*{Assunzioni}
  \begin{itemize}[nolistsep, noitemsep]
    \item Il programma termina dopo la prima chiamata alla funzione \texttt{pensoCheMaxSia} oppure allo scadere del tempo limite.
    \item $1 \le n \le 1\,000\,000$.
  \end{itemize}


  % Cosa sottomettere
  \section*{Cosa deve contenere il File da Sottomettere, e come viene gestito dal server}

  L'estensione del file che sottometti, *.c, *.cpp oppure *.pas,
  determina se esso viene compilato col compilatore del C (gcc),
  del c++ (g++) oppure del PASCAL (fpc).
  Le opzioni di compilazione, specifiche al linguaggio,
  sono riportate nel dettaglio sulla pagina "Descrizione" del problema
  in modo che tu possa simulare esattamente in locale cosa succede sul server.
  I comandi di compilazione riportati assumono che il nome del file sottomesso (privato di estensione) coincida col nome del problema.

  Nel caso di un problema interattivo come questo,
  dove il tuo programma interagisce con altri da noi predisposti
  (il grader ed anche un header da includere), i comandi di compilazione riportati compilano ed assemblano i vari pezzi (il tuo programma, il grader, e l'header).
  Se vuoi testare il comportamento del tutto in locale,
  devi allora scaricarti il nostro grader e l'header che rendiamo disponibile tra gli allegati alla pagina del problema.

  Tra gli allegati potremmo inoltre fornire una soluzione di esempio che,
  pur essendo compilabile sia in locale che dal server (ove ad esso sottomessa),
  faccia pochi o nessun punto in quanto non significativa per il problema
  in questione.

  In questo caso l'esempio compilabile reso disponibile al sito \`e il seguente:
\begin{verbatim}
// problem: max_x, example of a solution file, Romeo Rizzi Jan 2015
#include "ourLibToPlay.h"
void individua(long int n, long int maxLies) {
  if( pesa(0,n-1) > 0)  pensoCheMaxSia(0); 
  else pensoCheMaxSia(n-1); 
}
\end{verbatim}

Ovviamente questa soluzione di esempio non risolver\`a molte istanze
ma spero serva ad illustrare come potete muovervi ed \`e comunque compilabile sia in locale (se nella stessa cartella collocate anche il grader e l'header resi disponibili alla pagina del problema) sia ove sottoposta al server.
  
  
\end{document}

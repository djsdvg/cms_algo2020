% Template per generare 

\documentclass[a4paper,11pt]{article}
\usepackage{verbatim}
\usepackage{lmodern}
\renewcommand*\familydefault{\sfdefault}
\usepackage[utf8]{inputenc}
\usepackage[T1]{fontenc}
\usepackage[italian]{babel}
\usepackage{indentfirst}
\usepackage{graphicx}
\usepackage{tikz}
\newcommand*\circled[1]{\tikz[baseline=(char.base)]{
		\node[shape=circle,draw,inner sep=2pt] (char) {#1};}}
% \usepackage[group-separator={\,}]{siunitx}
\usepackage[left=2cm, right=2cm, bottom=3cm]{geometry}
\frenchspacing

\newcommand{\num}[1]{#1}

% Macro varie...
\newcommand{\file}[1]{\texttt{#1}}
\renewcommand{\arraystretch}{1.3}
\newcommand{\esempio}[2]{
\noindent\begin{minipage}{\textwidth}
\begin{tabular}{|p{11cm}|p{5cm}|}
	\hline
	\textbf{File \file{input.txt}} & \textbf{File \file{output.txt}}\\
	\hline
	\tt \small #1 &
	\tt \small #2 \\
	\hline
\end{tabular}
\end{minipage}
}

% Dati del task
\newcommand{\gara}{algo2017}
\newcommand{\nome}{Edit distance tra stringhe}
\newcommand{\nomebreve}{edit\_distance}

\begin{document}
% Intestazione
\noindent{\Large \gara}
\vspace{0.5cm}

\noindent{\Huge \textbf \nome~(\texttt{\nomebreve})}

% Descrizione del task
\section*{Descrizione del problema}
Ricevete in input due stringhe $s=s_1s_2 \ldots s_n$
e $t=t_1t_2 \ldots t_m$ sull'alfabeto $\{a,c,g,t\}$.

Dovete strasformare $s$ in $t$ attraverso una sequenza di operazioni quali:
\begin{description}
  \item[C, costo$\,=1$] cancellare un carattere dalla stringa $s$; 
  \item[A, costo$\,=0$] allineare un carattere di $s$ con un carattere della stringa $t$ ad esso identico.
  \item[I, costo$\,=1$] inserire un carattere dall'alfabeto $\{a,c,g,t\}$ entro la stringa $s$, per allinearlo con un carattere della stringa $t$ ad esso identico;
  \item[R, costo$\,=1$] rimpiazzare un carattere di $s$ con un carattere entro l`alfabeto $\{a,c,g,t\}$ da allineare con identico carattere in $t$.
\end{description}


% Input
\section*{File di input}
Da un file di nome \file{input.txt},
il programma legge due righe con le due stringhe, ciascuna specificata da un numero (la sua lughezza) e dalla sequenza di caratteri.

% Output
\section*{File di output}
Il programma deve scrivere in un file di nome \file{output.txt}.
La prima riga di tale file deve contenere un unico numero intero:
la distanza tra $s$ e $t$ (ossia al minimo costo di una sequenza di operazioni che trasformi $s$ in $t$).  
Se il valore riportato su questa prima riga \`e corretto,
viene aggiudicato quantomeno un punteggio parziale.
Per ottenere punteggio pieno,
il programma scrive poi una riga per ogni operazione
applicata per trasformare $s$ in $t$,
facendo riferimento agli indici
dei caratteri di $s$ e/o di $t$ su cui di volta in volta opera.
L'ordine con cui si procede \`e da sinistra verso destra su ambo le stringhe;
quando si cancella un elemento di $s$ se ne specifica l'indice entro la stringa $s$ originaria, quando si inserisce un elemento in $s$ si specifica l'indice dell'elemento di $t$ con cui esso vada ad allinearsi. La stringa $t$ non subisce modifiche, tutte le operazioni agiscono su $s$.

Se pu\`o aiutare proviamo a ridirla: Una stringa \`e una sequenza i cui elementi sono caratteri. Si pensi di avere un ditino indice che inizialmente punta al primo elemento di $s$, ed uno che punta al primo elemento di $t$.
E si pensi di stare componendo la $s$ come a valle della sua trasformazione in $t$. Ad ogni elemento (di $s$ o di $t$) resti comunque associata la sua posizione originaria entro la stringa di appartenenza (per altro $t$ non viene modificata).
Si dettagli allora se il carattere corrente di $s$ vada eliminato (spostandosi al successivo in $s$), oppure se inserire il carattere correntemente puntato in $t$ (spostandosi al successivo in $t$),
oppure se allineare i due caratteri (avanzando entrambi gli indici).
Ogni cancellazione ed ogni inserimento costano uno, mentre l'allineamento costa uno solo ove i due caratteri differiscano (ossia ove il carattere puntato in $s$ debba essere rimpiazzato col carattere puntato in $t$).

% Assunzioni
\section*{Assunzioni}
\begin{itemize}
\item $0 < M,N \le 100\,000$
\end{itemize}

% Subtasks
\section*{Subtask}

In tutti i subtask, met\`a del punteggio verr\`a aggiudicato
gi\`a con la sola determinazione della distanza da $s$ a $t$.
L'ultimo subtask \`e caratterizzato dalla presenza di un limite superiore
sulla distanza tra $s$ e $t$. Sconsigliamo di farlo se non per curiosit\`a personale (per altro d\`a zero punti).

\begin{itemize}
\item \textbf{Subtask 1 [0 punti]:} casi di esempio.
\item \textbf{Subtask 2 [25 punti]:} $M,N \le 10$.
\item \textbf{Subtask 3 [35 punti]:} $M,N \le 250$.
\item \textbf{Subtask 4 [40 punti]:} $M,N \leq 5\,000$.
\item \textbf{Subtask 5 [0 punti]:} $M,N \leq 100\,000$, $d(s,t) \leq 10$.
\end{itemize}


% Esempi
\section*{Esempi di input/output}

\esempio{
\verbatiminput{input0.txt}
}{
\verbatiminput{output0.txt}
}

\esempio{
\verbatiminput{input1.txt}
}{
\verbatiminput{output1.txt}
}

\esempio{
\verbatiminput{input2.txt}
}{
\verbatiminput{output2.txt}
}

\end{document}



\documentclass[a4paper,11pt]{article}
\usepackage{nopageno} % visto che in questo caso abbiamo una pagina sola
\usepackage{lmodern}
\renewcommand*\familydefault{\sfdefault}
\usepackage{sfmath}
\usepackage[utf8]{inputenc}
\usepackage[T1]{fontenc}
\usepackage[italian]{babel}
\usepackage{indentfirst}
\usepackage{graphicx}
\usepackage{tikz}
\usepackage{wrapfig}
\newcommand*\circled[1]{\tikz[baseline=(char.base)]{
		\node[shape=circle,draw,inner sep=2pt] (char) {#1};}}
\usepackage{enumitem}
% \usepackage[group-separator={\,}]{siunitx}
\usepackage[left=2cm, right=2cm, bottom=2cm]{geometry}
\frenchspacing

\newcommand{\num}[1]{#1}

% Macro varie...
\newcommand{\file}[1]{\texttt{#1}}
\renewcommand{\arraystretch}{1.3}
\newcommand{\esempio}[2]{
\noindent\begin{minipage}{\textwidth}
\begin{tabular}{|p{11cm}|p{5cm}|}
	\hline
	\textbf{File \file{input.txt}} & \textbf{File \file{output.txt}}\\
	\hline
	\tt \small #1 &
	\tt \small #2 \\
	\hline
\end{tabular}
\end{minipage}
}

\newcommand{\sezionetesto}[1]{
    \section*{#1}
}

\newcommand{\gara}{Esercizio per BaseCamp Marzo 2018}

%%%%% I seguenti campi verranno sovrascritti dall'\include{nomebreve} %%%%%
\newcommand{\nomebreve}{}
\newcommand{\titolo}{}

% Modificare a proprio piacimento:
\newcommand{\introduzione}{
%    \noindent{\Large \gara{}}
%    \vspace{0.5cm}
    \noindent{\Huge \textbf \titolo{}~(\texttt{\nomebreve{}})}
    \vspace{0.2cm}\\
}

\begin{document}

\renewcommand{\nomebreve}{bungeejumping}
\renewcommand{\titolo}{Jump Jump}

\introduzione{}

Un jumper deve portarsi dalla prima cella $v[0]$ di un vettore $v$ all'ultima cella $v[N-1]$, tramite una sequenza di salti.
I salti avvengono secondo le seguenti regole:
Quando il jumper di trova nella cella $i$-esima del vettore ($0\leq i \leq N-1$)
egli può portarsi nella cella $i\pm (v[i] - r)$ correndo un rischio di gravità $r\geq 0$. La compagnia di assicurazioni gli chiede un premio pari al massimo valore di rischio $r$ incorso da un salto lungo il percorso stabilito (classe di rischio).
Aiuta il jumper ad individuare un percorso che mantenga il rischio il più basso possibile.




\sezionetesto{Dati di input}
La prima riga del file \verb'input.txt' contiene un interi positivo $N$, la lunghezza del vettore.
La seconda riga del file contiene gli $N$ valori del vettore, nell'ordine, e separati da spazi.
Si vedano i due esempi.

\sezionetesto{Dati di output}
Nel file \verb'output.txt' si scriva un'unica riga contenente
un unico numero naturale: il minimo premio possibile da pagare per la polizza.\\


% Esempi
\sezionetesto{Esempio di input/output}
\esempio{
7

3 3 3 3 3 3 3
}{0}

\esempio{
8

3 3 3 3 3 3 3 3
}{1}


% Assunzioni
\sezionetesto{Assunzioni e note}
\begin{itemize}[nolistsep, noitemsep]
  \item $1 \le N \le 500$.
\end{itemize}
  
  \section*{Subtask}
  \begin{itemize}
    \item \textbf{Subtask 1 [0 punti]:} i due esempi del testo.
    \item \textbf{Subtask 2 [20 punti]:} $N \leq 10$.
    \item \textbf{Subtask 3 [40 punti]:} $v[i] \leq 100$ per ogni $i = 0,1,\ldots, N-1$.
    \item \textbf{Subtask 4 [40 punti]:} nessuna restrizione.
  \end{itemize}
  


\end{document}

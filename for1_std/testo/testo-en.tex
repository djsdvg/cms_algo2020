\documentclass[a4paper,11pt]{article}
\usepackage{nopageno} % visto che in questo caso abbiamo una pagina sola
\usepackage{lmodern}
\renewcommand*\familydefault{\sfdefault}
\usepackage{sfmath}
\usepackage[utf8]{inputenc}
\usepackage[T1]{fontenc}
\usepackage[italian]{babel}
\usepackage{indentfirst}
\usepackage{graphicx}
\usepackage{tikz}
\usepackage{wrapfig}
\newcommand*\circled[1]{\tikz[baseline=(char.base)]{
		\node[shape=circle,draw,inner sep=2pt] (char) {#1};}}
\usepackage{enumitem}
% \usepackage[group-separator={\,}]{siunitx}
\usepackage[left=2cm, right=2cm, bottom=3cm]{geometry}
\frenchspacing

\newcommand{\num}[1]{#1}

% Macro varie...
\newcommand{\file}[1]{\texttt{#1}}
\renewcommand{\arraystretch}{1.3}
\newcommand{\esempio}[2]{
\noindent\begin{minipage}{\textwidth}
\begin{tabular}{|p{11cm}|p{5cm}|}
	\hline
        \textbf{\file{input (da stdin)}} & \textbf{\file{output (su stdout)}}\\
%	\textbf{File \file{input.txt}} & \textbf{File \file{output.txt}}\\
	\hline
	\tt \small #1 &
	\tt \small #2 \\
	\hline
\end{tabular}
\end{minipage}
}

\newcommand{\sezionetesto}[1]{
    \section*{#1}
}

\newcommand{\gara}{Corso Tandem 2015 - prima edizione}

%%%%% I seguenti campi verranno sovrascritti dall'\include{nomebreve} %%%%%
\newcommand{\nomebreve}{}
\newcommand{\titolo}{}

% Modificare a proprio piacimento:
\newcommand{\introduzione}{
%    \noindent{\Large \gara{}}
%    \vspace{0.5cm}
    \noindent{\Huge \textbf \titolo{}~(\texttt{\nomebreve{}})}
    \vspace{0.2cm}\\
}

\begin{document}

\renewcommand{\nomebreve}{for1\_std}
\renewcommand{\titolo}{First exercise on {\tt for} - emulate the {\tt seq} command}

\mbox{\ }
\vspace{-1.6cm}


\introduzione{}

You receive in input a positive natural $N$ and,
in output, you must yield the sorted sequence of the first $N$ natural numbers:\\

  $1$  \ \ $2$  \ \ $3$  \ \ $4$  \ \ $\cdots$  \ \ $N-2$  \ \ $N-1$  \ \ $N$

You can separate the $N$ output numbers with spaces or, if you prefer,
with new lines, so that each number will go on a different line and you will be perfectly emulating the \verb'seq' command of the \verb'bash' shell.
Those of you who have seen some python will rather think of:

\verb'print range(1,N+1)'

However, whatever the chosen programming language, try to get the job done by means of a for cycle. 

\sezionetesto{input description}
Your only input, got from \verb'stdin', is a positive integer number $N$.

\sezionetesto{output description}
You program should output on \verb'stdout' one single line containing the sequence\\

  $1$  \ \ $2$  \ \ $3$  \ \ $4$  \ \ $\cdots$  \ \ $N-2$  \ \ $N-1$  \ \ $N$

comprising the first $N$ positive natural number in their order.
Any two consecutive numbers should be separated by spaces.


% Esempi
\sezionetesto{input/output example}
\esempio{6}{1 2 3 4 5 6}
\esempio{10}{1 2 3 4 5 6 7 8 9 10}

% Assunzioni
\sezionetesto{Assumptions}
\begin{itemize}[nolistsep, noitemsep]
  \item $1 \le N \le 10\,000$.
\end{itemize}
  
  \section*{Subtasks}
  \begin{itemize}
    \item \textbf{Subtask 0 [10 punti]:} the two examples above.
    \item \textbf{Subtask 1 [15 punti]:} $N = 17$.
    \item \textbf{Subtask 2 [25 punti]:} $N \leq 20$.
    \item \textbf{Subtask 4 [50 punti]:} no special restriction.
  \end{itemize}
  


\end{document}

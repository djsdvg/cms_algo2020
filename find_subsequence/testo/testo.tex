\documentclass[a4paper,11pt]{article}
\usepackage{nopageno} % visto che in questo caso abbiamo una pagina sola
\usepackage{lmodern}
\renewcommand*\familydefault{\sfdefault}
\usepackage{sfmath}
\usepackage[utf8]{inputenc}
\usepackage[T1]{fontenc}
\usepackage[italian]{babel}
\usepackage{indentfirst}
\usepackage{graphicx}
\usepackage{tikz}
\usepackage{wrapfig}
\newcommand*\circled[1]{\tikz[baseline=(char.base)]{
		\node[shape=circle,draw,inner sep=2pt] (char) {#1};}}
\usepackage{enumitem}
% \usepackage[group-separator={\,}]{siunitx}
\usepackage[left=2cm, right=2cm, bottom=2cm]{geometry}
\frenchspacing

\newcommand{\num}[1]{#1}

% Macro varie...
\newcommand{\file}[1]{\texttt{#1}}
\renewcommand{\arraystretch}{1.3}
\newcommand{\esempio}[2]{
\noindent\begin{minipage}{\textwidth}
\begin{tabular}{|p{11cm}|p{5cm}|}
	\hline
	\textbf{File \file{input.txt}} & \textbf{File \file{output.txt}}\\
	\hline
	\tt \small #1 &
	\tt \small #2 \\
	\hline
\end{tabular}
\end{minipage}
}

\newcommand{\sezionetesto}[1]{
    \section*{#1}
}

\newcommand{\gara}{Esame algoritmi 2019-02-26 VR}

%%%%% I seguenti campi verranno sovrascritti dall'\include{nomebreve} %%%%%
\newcommand{\nomebreve}{}
\newcommand{\titolo}{}

% Modificare a proprio piacimento:
\newcommand{\introduzione}{
    \noindent{\Large \gara{}}\\
    \vspace{0.5cm}
    \noindent{\Huge \textbf \titolo{}~(\texttt{\nomebreve{}})}
    \vspace{0.2cm}\\
}

\begin{document}

\makeatletter
\renewcommand{\this@inputfilename}{\texttt{stdin}}
\renewcommand{\this@outputfilename}{\texttt{stdout}}
\makeatother

Una stringa è una sequenza finita di caratteri presi da un qualche alfabeto di riferimento. 
Ad esempio, $s=GATTA$ è una stringa sull'alfabeto $\{A,C,G,T\}$, come anche sull'alfabeto $\{A,G,T\}$.
Indichiamo con $len(s)$ il numero di caratteri della stringa~$s$, in questo caso $5$.
Si noti che la stringa $s$ appare come sottosequenza dentro la stringa $t=TGGTTCAAGTCGTCACA$, 
cosa che quì preferiamo mettere in evidenza tramite opportuna sottolineatura:\\

\renewcommand{\arraystretch}{0.2}
\begin{tabular}{l||c|c|c|c|c|c|c|c|c|c|c|c|c|c|c|c|c|}
  indice di posizione       &  & & & & & & & & & & 1&1&1&1&1&1&1 \\
                            & 0&1&2&3&4&5&6&7&8&9& 0&1&2&3&4&5&6 \\
  \hline
                            &  & & & & & & & & & &  & & & & & &  \\
                            &  & & & & & & & & & &  & & & & & &  \\  
  stringa testo $t$         & T&$\underline{G}$&G&T&T&C&$\underline{A}$&A&G&$\underline{T}$&C&G&$\underline{T}$&C&$\underline{A}$&C&A \\
                            &  & & & & & & & & & &  & & & & & &  \\
                            &  & & & & & & & & & &  & & & & & &  \\
                            &  & & & & & & & & & &  & & & & & &  \\  
  prima occorrenza di $s$ in $t$  &-&G&-&-&-&-&A&-&-&T&-&-&T&-&A&-&- \\  
\end{tabular}  
\renewcommand{\arraystretch}{1.2}  
\bigskip

In realtà gli ultimi 2 caratteri del testo $t$ non sono nemmeno serviti, la stringa 
$s$ appare già come sottosequenza del prefisso $t_{15}=TGGTTCAAGTCGTCA$ di $t$ di 
lunghezza $15$ (la sequenza di caratteri dalla posizione $0$ alla posizione $14$).
Data una stringa $s$ ed una stringa testo $t$, ti chiediamo di trovare il più breve 
prefisso di $t$ che contenga $s$ come sottosequenza.
Se $s$ non è una sottosequenza di $t$ si ritorni allora il numero $-1$.

\bigskip
\noindent
Implementa il tuo metodo per fare questo entro la seguente funzione:

\begin{verbatim}
def find_subsequence(s, t):  # questa è la funzione che devi perfezionare
    risp = -1   # risposta corretta quando la stringa s non è sottosequenza della stringa t
    # inserisci quì la tua logica/metodo per risolvere il problema
    return risp
\end{verbatim}

Trovi un template della soluzione nel file \textbf{find-subsequence\_template\_sol.py}, 
dovrai solo risistemare l'implementazione della funzione richiesta che al momento soddisfa 
alla consegna solo per quelle istanze in cui la prima sequenza passatale non è sottosequenza della seconda.

\InputFile
Il vostro programma riceve in input una stringa~$s$ sull'alfabeto $\{A,C,G,T\}$ e, 
su riga successiva, una stringa~$t$ sempre sullo stesso alfabeto. 

\OutputFile
Restituisci il numero $-1$ se la data stringa $s$ non è sottosequenza della stringa 
$t$ assegnata, altrimenti restituisci il più piccolo numero naturale $i$ tale che 
che $s$ è sottosequenza del prefisso $t_i$ dei soli primi $i$ caratteri di $t$.

\textbf{NOTA}: viene fornita una descrizione del formato di input/output soltanto 
per facilitarvi il test sul vostro computer. Per sottomettere il problema è obbligatorio 
utilizzare il template che potete scaricare fra gli allegati del problema, avendo 
cura di modificare solamente l'implementazione delle funzioni richieste. Questo 
è necessario per garantire la compatibilità del vostro programma con il sistema 
di valutazione, che potrebbe utilizzare una versione di python diversa (quale python2).

\Examples
\begin{example}
\exmpfile{find-subsequence.input0.txt}{find-subsequence.output0.txt}%
\exmpfile{find-subsequence.input1.txt}{find-subsequence.output1.txt}%
\exmpfile{find-subsequence.input2.txt}{find-subsequence.output2.txt}%
\end{example}

\Constraints
\begin{itemize}[nolistsep, noitemsep]
  \item $1\leq len(p) \leq len(t)$
\end{itemize}

\Scoring
  \begin{itemize}
    \item \textbf{Subtask 1 [0 punti]:} gli esempi del testo.
    \item \textbf{Subtask 2 [20 punti]:} $len(p) = len(t)$.
    \item \textbf{Subtask 3 [20 punti]:} $len(p) = 1$.
    \item \textbf{Subtask 4 [20 punti]:} $len(p) \le 2$.
    \item \textbf{Subtask 5 [20 punti]:} $len(t) \le 10$.
    \item \textbf{Subtask 6 [20 punti]:} $len(t) \le 1000$.
    \item \textbf{Subtask 7 [0 punti]:} $len(t) \le 1\,000\,000$.
  \end{itemize}


\end{document}

% Template per generare

\documentclass[a4paper,11pt]{article}
\usepackage{lmodern}
\renewcommand*\familydefault{\sfdefault}
\usepackage{sfmath}
\usepackage[utf8]{inputenc}
\usepackage[T1]{fontenc}
\usepackage[italian]{babel}
\usepackage{indentfirst}
\usepackage{graphicx}
\usepackage{tikz}
\newcommand*\circled[1]{\tikz[baseline=(char.base)]{
    \node[shape=circle,draw,inner sep=2pt] (char) {#1};}}
\usepackage{enumitem}
% \usepackage[group-separator={\,}]{siunitx}
\usepackage[left=2cm, right=2cm, bottom=3cm]{geometry}
\frenchspacing

\newcommand{\num}[1]{#1}

% Macro varie...
\newcommand{\file}[1]{\texttt{#1}}
\renewcommand{\arraystretch}{1.3}
\newcommand{\esempio}[2]{
  \noindent\begin{minipage}{\textwidth}
    \begin{tabular}{|p{11cm}|p{5cm}|}
      \hline
      \textbf{File \file{input.txt}} & \textbf{File \file{output.txt}}\\
      \hline
      \tt \small #1 &
      \tt \small #2 \\
      \hline
    \end{tabular}
  \end{minipage}
}

% Dati del task
\newcommand{\nome}{Trova le affinità di.. stringa}
\newcommand{\nomebreve}{string\_matching}

\begin{document}
   
  % Intestazione  
  \noindent{\Huge \textbf \nome~(\texttt{\nomebreve})}
  \vspace{0.2cm}\\
  
  % Descrizione del task
\section*{Descrizione del problema}
\noindent
Supponi che ti sia data una stringa \textsc{P} di caratteri appartenenti all'alfabeto \textsc{A}, di lunghezza $n$. Tale stringa prende il nome di ``pattern''. Data una stringa nascosta \textsc{T}, di lunghezza nota $m$, il tuo ruolo è quello di trovare se e quante volte la stringa \textsc{P} è presente in \textsc{T}, utilizzando il minor numero di confronti tra caratteri.

\medskip
\noindent
Nel nostro caso particolare, \textsc{A} è l'alfabeto inglese di 26 lettere minuscole ('a', 'b', 'c', ...). 
\noindent
Ricorda che devi trovare \textbf{TUTTI} i matching di \textsc{P} in \textsc{T}.

\medskip
\noindent
Per trovare i matching tra \textsc{P} e la stringa segreta \textsc{T}, devi scrivere una procedura che riporti quanti matching vi sono tra le due stringhe e, per ognuno di essi, in quale posizione si trova.

\medskip
\noindent
Il file da sottomettere deve avere la seguente struttura:
\begin{verbatim}
#include "ourLibToPlay.h"

void solve(char *pattern, int patt_len, int text_len) {
   ...
} 
\end{verbatim}

\noindent
dove, \texttt{pattern} è la stinga \textsc{P}, \texttt{patt\_len} è l'intero $n$ e \texttt{text\_len} è l'intero $m$.


\bigskip\medskip
\noindent
Per confrontare il pattern con la stringa nascosta, potrai servirti della seguente funzione:

\medskip  
\noindent
\texttt{int charCompare(char p, int pos\_in\_txt)}
\medskip
  
\noindent
La funzione \texttt{charCompare} richiede in input:

\begin{itemize}
\item il carattere \texttt{p} del pattern da confrontare;
\item la posizione \texttt{pos\_in\_txt} nella stringa nascosta dove si trova il carattere con il quale vuoi eseguire il confronto.
\end{itemize}  

\noindent La funzione \texttt{charCompare} ritorna $0$ se i due caratteri sono diversi, un numero maggiore di $0$ altrimenti.

\bigskip
\medskip
\noindent
Vi sono diversi subtask, divisi a seconda del numero massimo di confronti a tua disposizione. 
  
\medskip
\noindent
Una volta individuato senza alcun margine di dubbio il numero di matching trovati e la loro posizione nella stringa \textsc{T}, devi consegnarlo invocando la funzione:
  
\medskip
  
\noindent
\texttt{void pensoCheMatchingSia(int risposta [ ], int risp\_len)}

\medskip
dove l'array \texttt{risposta [ ]} contiene tutte le posizioni nella stringa \textsc{T} dove comincia il match con il tuo pattern e l'intero \texttt{risp\_len} contiene il numero di matching trovati.
  
\section*{Subtask}
\begin{itemize}
\item \textbf{Subtask 0 [20 punti]:} nessuna limitazione sul numero di confronti.
\item \textbf{Subtask 1 [40 punti]:} hai a disposizione al pi\`u $3\times m$ confronti.
\item \textbf{Subtask 2 [40 punti]:} hai a disposizione al pi\`u $4/3\times m$ confronti.
\end{itemize}
  
  % Assunzioni
  \section*{Assunzioni}
  \begin{itemize}[nolistsep, noitemsep]
    \item Il programma termina dopo la prima chiamata alla funzione \texttt{pensoCheMatchingSia} oppure allo scadere del tempo limite.
    \item Tempo Limite: 1 sec.
    \item Spazio Limite: 256 MB.
    \item $ 10 \leq m \leq 100000 $.
    \item $ 1 \leq n \leq 100$ 
    \item $ n \leq m$
  \end{itemize}


  % Cosa sottomettere
  \section*{Cosa deve contenere il File da Sottomettere, e come viene gestito dal server}

\noindent
  L'estensione del file che sottometti, *.c oppure *.cpp,
  determina se esso viene compilato col compilatore del C (gcc)   oppure del c++ (g++).
  Le opzioni di compilazione, specifiche al linguaggio,
  sono riportate nel dettaglio sulla pagina "Descrizione" del problema
  in modo che tu possa simulare esattamente in locale cosa succede sul server.
  I comandi di compilazione riportati assumono che il nome del file sottomesso (privato di estensione) coincida col nome del problema.

\noindent
  Nel caso di un problema interattivo come questo,
  dove il tuo programma interagisce con altri da noi predisposti
  (il grader ed anche un header da includere), i comandi di compilazione riportati compilano ed assemblano i vari pezzi (il tuo programma, il grader, e l'header).
  Se vuoi testare il comportamento del tutto in locale,
  devi allora scaricarti il nostro grader e l'header che rendiamo disponibile tra gli allegati alla pagina del problema.
  
\noindent
  Tra gli allegati potremmo inoltre fornire una soluzione di esempio che,
  pur essendo compilabile sia in locale che dal server (ove ad esso sottomessa),
  faccia pochi o nessun punto in quanto non significativa per il problema
  in questione.
  
\noindent
  In questo caso l'esempio compilabile reso disponibile al sito \`e il seguente:
\begin{verbatim}
// problem: string_matching, example of a dumb solution file

#include "ourLibToPlay.h"

void solve(char *pattern, int patt_len, int text_len) {
    int sol[1];
    if(charCompare(pattern[0], 0)) {
        sol[0] = 0;
        pensoCheMatchingSia(sol, 1);    
    } else {
        sol[0] = 1;    
        pensoCheMatchingSia(sol, 1);    
    }
}
\end{verbatim}

\noindent
Ovviamente questa soluzione di esempio non risolver\`a molte istanze
ma spero serva ad illustrare come potete muovervi ed \`e comunque compilabile sia in locale (se nella stessa cartella collocate anche il grader e l'header resi disponibili alla pagina del problema) sia ove sottoposta al server.
  
  
\end{document}

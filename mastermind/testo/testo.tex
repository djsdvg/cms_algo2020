% Template per generare

\documentclass[a4paper,11pt]{article}
\usepackage{lmodern}
\renewcommand*\familydefault{\sfdefault}
\usepackage{sfmath}
\usepackage[utf8]{inputenc}
\usepackage[T1]{fontenc}
\usepackage[italian]{babel}
\usepackage{indentfirst}
\usepackage{graphicx}
\usepackage{tikz}
\newcommand*\circled[1]{\tikz[baseline=(char.base)]{
    \node[shape=circle,draw,inner sep=2pt] (char) {#1};}}
\usepackage{enumitem}
% \usepackage[group-separator={\,}]{siunitx}
\usepackage[left=2cm, right=2cm, bottom=3cm]{geometry}
\frenchspacing

\newcommand{\num}[1]{#1}

% Macro varie...
\newcommand{\file}[1]{\texttt{#1}}
\renewcommand{\arraystretch}{1.3}
\newcommand{\esempio}[2]{
  \noindent\begin{minipage}{\textwidth}
    \begin{tabular}{|p{11cm}|p{5cm}|}
      \hline
      \textbf{File \file{input.txt}} & \textbf{File \file{output.txt}}\\
      \hline
      \tt \small #1 &
      \tt \small #2 \\
      \hline
    \end{tabular}
  \end{minipage}
}

% Dati del task
\newcommand{\nome}{Indovina il codice segreto!}
\newcommand{\nomebreve}{mastermind}

\begin{document}
  
  
  % Intestazione  
  \noindent{\Huge \textbf \nome~(\texttt{\nomebreve})}
  \vspace{0.2cm}\\
  
  % Descrizione del task
  \section*{Descrizione del problema}
  \noindent
Immaginati nelle vesti di un ``decodificatore''. 
Il tuo avversario, un ``codificatore'', ha scritto un codice segreto di quattro cifre, ognuna delle quali può assumere un valore tra $0$ e $5$.

\noindent
Per sconfiggere il tuo avversario, devi scrivere una procedura che individui il suo codice segreto, effettuando il minor numero di tentativi.
  Il file da sottomettere deve avere la seguente struttura:
\begin{verbatim}
#include "ourLibToPlay.h"

void solve() {
   ...
} 
\end{verbatim}

\noindent

\medskip
\noindent
Durante la fase di gioco, per capire quanto il tuo codice si avvicina a quello del tuo avversario, potrai servirti della seguente funzione:

\medskip  
\noindent
\texttt{void checkCode(int attempt[ ], int result[ ])}
  
\medskip
  
\noindent
La funzione \texttt{checkCode} richiede in input:

\begin{itemize}
\item l'array \texttt{attempt[ ]} di quattro cifre contenente il codice da te proposto;
\item un array \texttt{result[ ]} di due elementi in cui verr\`a salvato il risultato.
\end{itemize}  

\noindent
Il primo elemento di \texttt{result[ ]} indica il \textbf{numero di cifre giuste al posto giusto} che hai indovinato, mentre il secondo elemento di \texttt{result[ ]} indica \textbf{il numero di cifre giuste al posto sbagliato} che hai indovinato, ovvero quelle cifre che sono presenti nel codice del tuo avversario, ma non nella posizione da te proposta.

\medskip
\noindent Ricorda che il tentativo che ti porta alla soluzione corretta, ovvero a scoprire il codice del tuo avversario, \textbf{NON} è conteggiato nel massimo numero di tentativi a tua disposizione. Ad esempio, se per scoprire il codice del tuo avversario hai 10 tentativi, significa che puoi indovinare fino a 10 codici \textbf{errati} prima di arrivare alla soluzione corretta.

\medskip
\noindent
Vi sono diversi subtask, divisi a seconda del numero massimo di tentativi a tua disposizione. 
  
\medskip
\noindent
Una volta individuato senza alcun margine di dubbio il codice del tuo avversario, devi consegnarlo invocando la funzione:
  
\medskip
  
\noindent
\texttt{void pensoCheCodiceSia(int risposta [ ])}

\medskip
\noindent
dove l'array \texttt{risposta [ ]} contiene il codice di Alice da te scoperto.
  
\section*{Subtask}
\begin{itemize}
\item \textbf{Subtask 0 [20 punti]:} nessuna limitazione sul numero di tentativi.
\item \textbf{Subtask 1 [40 punti]:} hai a disposizione al pi\`u 10 tentativi.
\item \textbf{Subtask 2 [40 punti]:} hai a disposizione al pi\`u 6 tentativi.
\end{itemize}
  
  % Assunzioni
  \section*{Assunzioni}
  \begin{itemize}[nolistsep, noitemsep]
    \item Il programma termina dopo la prima chiamata alla funzione \texttt{pensoCheCodiceSia} oppure allo scadere del tempo limite.
    \item Tempo Limite: 1 sec.
    \item Spazio Limite: 256 MB.
  \end{itemize}


  % Cosa sottomettere
  \section*{Cosa deve contenere il File da Sottomettere, e come viene gestito dal server}

\noindent
  L'estensione del file che sottometti, *.c oppure *.cpp,
  determina se esso viene compilato col compilatore del C (gcc)   oppure del c++ (g++).
  Le opzioni di compilazione, specifiche al linguaggio,
  sono riportate nel dettaglio sulla pagina "Descrizione" del problema
  in modo che tu possa simulare esattamente in locale cosa succede sul server.
  I comandi di compilazione riportati assumono che il nome del file sottomesso (privato di estensione) coincida col nome del problema.

\noindent
  Nel caso di un problema interattivo come questo,
  dove il tuo programma interagisce con altri da noi predisposti
  (il grader ed anche un header da includere), i comandi di compilazione riportati compilano ed assemblano i vari pezzi (il tuo programma, il grader, e l'header).
  Se vuoi testare il comportamento del tutto in locale,
  devi allora scaricarti il nostro grader e l'header che rendiamo disponibile tra gli allegati alla pagina del problema.
  
\noindent
  Tra gli allegati potremmo inoltre fornire una soluzione di esempio che,
  pur essendo compilabile sia in locale che dal server (ove ad esso sottomessa),
  faccia pochi o nessun punto in quanto non significativa per il problema
  in questione.
  
\noindent
  In questo caso l'esempio compilabile reso disponibile al sito \`e il seguente:
\begin{verbatim}
// problem: mastermind, example of a dumb solution file

#include "ourLibToPlay.h"

void solve(){
    int sol[4] = {1, 4, 5, 0};
    int res[2];
    checkCode(sol, res);
    if(res[0]==4){
        pensoCheCodiceSia(sol);    
    } else {
        sol[0] = 1; sol[1] = 2; sol[2] = 3; sol[3] = 4;    
        pensoCheCodiceSia(sol);    
    }
}
\end{verbatim}

\noindent
Ovviamente questa soluzione di esempio non risolver\`a molte istanze
ma spero serva ad illustrare come potete muovervi ed \`e comunque compilabile sia in locale (se nella stessa cartella collocate anche il grader e l'header resi disponibili alla pagina del problema) sia ove sottoposta al server.
  
  
\end{document}

% Template per generare 

\documentclass[a4paper,11pt]{article}
\usepackage{lmodern}
\renewcommand*\familydefault{\sfdefault}
\usepackage[utf8]{inputenc}
\usepackage[T1]{fontenc}
\usepackage[italian]{babel}
\usepackage{indentfirst}
\usepackage{graphicx}
\usepackage{tikz}
\newcommand*\circled[1]{\tikz[baseline=(char.base)]{
		\node[shape=circle,draw,inner sep=2pt] (char) {#1};}}
% \usepackage[group-separator={\,}]{siunitx}
\usepackage[left=2cm, right=2cm, bottom=3cm]{geometry}
\frenchspacing

\newcommand{\num}[1]{#1}

% Macro varie...
\newcommand{\file}[1]{\texttt{#1}}
\renewcommand{\arraystretch}{1.3}
\newcommand{\esempio}[2]{
\noindent\begin{minipage}{\textwidth}
\begin{tabular}{|p{11cm}|p{5cm}|}
	\hline
	\textbf{File \file{input.txt}} & \textbf{File \file{output.txt}}\\
	\hline
	\tt \small #1 &
	\tt \small #2 \\
	\hline
\end{tabular}
\end{minipage}
}

% Dati del task
\newcommand{\gara}{esercizio di introduzione alla programmazione dinamica}
\newcommand{\nome}{Campo minato}
\newcommand{\nomebreve}{minato}

\begin{document}
% Intestazione
\noindent{\Large \gara}
\vspace{0.5cm}

\noindent{\Huge \textbf \nome~(\texttt{\nomebreve})}

% Descrizione del task
\section*{Descrizione del problema}

Un robot deve portarsi dalla cella $(1,1)$
alla cella $(M,N)$ di una scacchiera rettangolare
composta da $M \times N$ celle di forma quadrata.
Il robot non può mai fuoriuscire dalla griglia.
Sono previsti due soli tipi di mosse:
\begin{description}
\item[scendi] il robot si muove dalla cella $(i,j)$ alla cella $(i+1,j)$. (Mossa disponibile solo quando $i<M$) 
\item[procedi] il robot si muove dalla cella $(i,j)$ alla cella $(i,j+1)$. (Mossa disponibile solo quando $j<N$)   
\end{description}  

Il terreno è minato: alcune delle celle sono interdette e devono essere evitate, fortunatamente siamo provvisti di una mappa che precisa
la situazione di ogni cella $(i,j)$ della griglia:
sono da evitare le celle segnalate da una crocetta (\file{map}$[i,j]=$'+'),
le altre celle sono tutte transitabili e riportano invece un asterisco (\file{map}$[i,j]=$'*').
In particolare, \file{map}$[1,1]=$\file{map}$[M,N]=$'*'.


% Input
\section*{File di input}
Il programma deve leggere da un file di nome \file{input.txt}.
La prima riga contiene due interi positivi $M$ e $N$, separati da spazio, i quali rappresentano le dimensioni della scacchiera.
Le successive $M$ righe forniscono la mappa, ossia:
l'$i$-esima di queste $M$ righe contiene una sequenza di $N$ caratteri
ciascuno dei quali è un '+' (cella minata, ossia interdetta) oppure un '*' (cella transitabile).
Il $j$-esimo di questi caratteri è \file{map}$[i,j]$.
Nel file \file{input.txt}
questi caratteri NON sono separati da degli spazi:
le righe dalla~$2$ alla~$M+1$ constano di precisamente $N$ caratteri
cui aggiungere il fine riga.

% Output
\section*{File di output}
Il programma deve scrivere in un file di nome \file{output.txt}. Deve venire stampato un unico numero, il numero di percorsi diversi che il robot può seguire per andare dalla cella $(1,1)$ alla cella $(M,N)$.


% Esempi
\section*{Esempio di input/output}
\esempio{
5 4

****

+***

*+**

++**

++**
}{9}


% Assunzioni
\section*{Assunzioni}
\begin{itemize}
\item $0 < M,N \leq 200$
\item \file{map}$[1,1]=$\file{map}$[M,N]=$'*'..
\item Il risultato è interno all'intervallo rappresentato dagli int a $32$ bit.

\end{itemize}

% Subtasks
\section*{Subtask}
\begin{itemize}
\item \textbf{Subtask 1 [\phantom{1}0 punti]:} caso di esempio.
\item \textbf{Subtask 2 [10 punti]:} $M = 2$, $N \leq 200$, nessuna mina.
\item \textbf{Subtask 3 [10 punti]:} $M = 2$, $N \leq 200$.
\item \textbf{Subtask 4 [10 punti]:} $M = 3$, $N \leq 200$, nessuna mina.
\item \textbf{Subtask 5 [10 punti]:} $M = 3$, $N \leq 200$.
\item \textbf{Subtask 6 [10 punti]:} nessuna mina, $M,N \leq 10$.
\item \textbf{Subtask 7 [10 punti]:} nessuna mina, $M,N \leq 30$.
\item \textbf{Subtask 8 [10 punti]:} $M,N \leq 10$.
\item \textbf{Subtask 9 [10 punti]:} $M,N \leq 20$.
\item \textbf{Subtask 10 [10 punti]:} $M,N \leq 50$.
\item \textbf{Subtask 11 [10 punti]:} $M,N \leq 200$.  
\end{itemize}


\end{document}

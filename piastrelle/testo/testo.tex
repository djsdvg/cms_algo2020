% Template per generare 

\documentclass[a4paper,11pt]{article}
\usepackage{lmodern}
\renewcommand*\familydefault{\sfdefault}
\usepackage[utf8]{inputenc}
\usepackage[T1]{fontenc}
\usepackage[italian]{babel}
\usepackage{indentfirst}
\usepackage{graphicx}
\usepackage{tikz}
\newcommand*\circled[1]{\tikz[baseline=(char.base)]{
		\node[shape=circle,draw,inner sep=2pt] (char) {#1};}}
% \usepackage[group-separator={\,}]{siunitx}
\usepackage[left=2cm, right=2cm, bottom=3cm]{geometry}
\frenchspacing

\newcommand{\num}[1]{#1}

% Macro varie...
\newcommand{\file}[1]{\texttt{#1}}
\renewcommand{\arraystretch}{1.3}
\newcommand{\esempio}[2]{
\noindent\begin{minipage}{\textwidth}
\begin{tabular}{|p{11cm}|p{5cm}|}
	\hline
	\textbf{File \file{input.txt}} & \textbf{File \file{output.txt}}\\
	\hline
	\tt \small #1 &
	\tt \small #2 \\
	\hline
\end{tabular}
\end{minipage}
}

% Dati del task
\newcommand{\gara}{phd2015}
\newcommand{\nome}{Piastrellature}
\newcommand{\nomebreve}{piastrelle}

\begin{document}
% Intestazione
\noindent{\Large \gara}
\vspace{0.5cm}

\noindent{\Huge \textbf \nome~(\texttt{\nomebreve})}

% Descrizione del task
\section*{Descrizione del problema}
Pippo ha un corridoio di dimensione $1 \times N$ da piastellare. Lui ha a disposizione solo piastrelle di 
dimensioni $1 \times 1$ e $1 \times 2$. Essendo questo il corridoio dell'ingresso di casa sua, lui vorrebbe che fosse il 
più bello possibile, quindi ha bisogno di conoscere tutte le possibili disposizioni delle piastrelle che 
potrebbe usare per completare il lavoro.

% Input
\section*{File di input}
Il programma deve leggere da un file di nome \file{input.txt} nel quale è presente un unico intero $N$, la dimensione del corridoio.

% Output
\section*{File di output}
Il programma deve scrivere in un file di nome \file{output.txt}. In ognuna delle $K$ righe stampate deve essere presente una disposizione valida usando il seguente formato:
\begin{itemize}
\item Le piastrelle $1 \times 1$ vengono rappresentate con \texttt{[O]} (lettera `o' maiuscola)
\item Le piastrelle $1 \times 2$ vengono rappresentate con \texttt{[OOOO]} (quadrupla `o' maiuscola)
\item Le piastrelle non vengono separate da alcun carattere
\item Le righe devono essere in ordine lessicografico: prima quelle che iniziano con $1 \times 1$ poi quelle con $1 \times 2$
\end{itemize}

% Assunzioni
\section*{Assunzioni}

\begin{itemize}
\item $1 \le N \le 25$
\end{itemize}

% Subtasks
\section*{Subtask}
\begin{itemize}
\item \textbf{Subtask 1 [\phantom{1}5 punti]:} caso di esempio.
\item \textbf{Subtask 2 [\phantom{1}5 punti]:} $N \le 5$.
\item \textbf{Subtask 3 [30 punti]:} $N \le 10$.
\item \textbf{Subtask 4 [30 punti]:} $N \le 15$.
\item \textbf{Subtask 5 [30 punti]:} nessuna limitazione specifica.
\end{itemize}

% Esempi
\section*{Esempio di input/output}
\esempio{4}{
[O][O][O][O]

[O][O][OOOO]

[O][OOOO][O]

[OOOO][O][O]

[OOOO][OOOO]
}

\end{document}

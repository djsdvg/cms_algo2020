
\documentclass[a4paper,11pt]{article}

\usepackage[utf8x]{inputenc}
\SetUnicodeOption{mathletters}
\SetUnicodeOption{autogenerated}

\usepackage[italian]{babel}
\usepackage{booktabs}
\usepackage{mathpazo}
\usepackage{graphicx}
\usepackage[left=2cm, right=2cm, bottom=3cm]{geometry}
\frenchspacing

\begin{document}
\noindent {\Large Primo stage (Volterra) - Allenamento primo stage}
\vspace{0.5cm}

\noindent {\Huge mst (\texttt{mst})}


\vspace{0.5cm}
\noindent {\Large Difficoltà D = 1 (tempo limite 2 sec).}

\section*{Descrizione del problema}
  
    Si calcoli il minimo albero ricoprente di un grafo pesato non
    orientato avente $N$ nodi e $M$ archi.
  

\section*{Dati di input}
  
    Il file input.txt contiene sulla prima riga $N$
    e $M$, separati da uno spazio. Le successive $M$
    righe contengono tre numeri interi
    positivi, $u$, $v$, $w$, che
    rappresentano un arco di peso $w$ che collega i
    nodi $u$ e $v$ (numerati da 1 a $N$).
  

\section*{Dati di output}
  
    Il file output.txt contiene sulla prima riga il peso dell'albero e
    sulle successive $N-1$ righe le coppie $u,v$
    (separate da uno spazio) che rappresentano gli archi dell'albero.
  
  \section*{Assunzioni}
  \begin{itemize}
  
    \item $1 ≤ N ≤ 10000$
    \item $2 ≤ M ≤ 1000000$
    \item $1 ≤ Wi ≤ 2^{42}$
    \item Il peso dell'MST è sicuramente $≤ 2^{58}$
  \end{itemize}

\section*{Esempi di input/output}

  
    \noindent
    \begin{tabular}{p{11cm}|p{5cm}}
    \toprule
    \textbf{File \texttt{input.txt}}
    & \textbf{File \texttt{output.txt}}
    \\
    \midrule
    \scriptsize
    \begin{verbatim}
7 9
1 2 7
2 3 21
1 3 14
1 4 30
4 3 10
3 5 1
5 6 6
5 7 9
6 7 4
    \end{verbatim}
    &
    \scriptsize
    \begin{verbatim}
42
1 2
1 3
3 4
3 5
5 6
6 7
    \end{verbatim}
    \\
    \bottomrule
    \end{tabular}
  


\end{document}

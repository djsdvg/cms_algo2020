\renewcommand{\nomebreve}{bonsai}
\renewcommand{\titolo}{Task 3}


\introduzione{}
\noindent {\Large Difficolt\`a� D = 1}
\vspace{0.5cm}





\textbf{\large{IL BONSAI DEL MAESTRO MIYAGI}}\\
\smallskip

Il Maestro Miyagi \`e conosciuto per i suoi inusuali insegnamenti: un giorno decide di insegnare al suo apprendista, Daniel-san, come raggiungere la pace interiore necessaria per eseguire correttamente la leggendaria ``Tecnica della Gru''.
Come esercizio Daniel-san dovr\`a potare il vecchio bonsai del suo Maestro che tiene gelosamente in casa propria.
Il Maestro Miyagi mostra a Daniel-san una foto di un bonsai trovato nel mensile de ``Il tagliaerba'' chiedendo al suo allievo di potare il vecchio bonsai in modo tale da ottenere un bonsai uguale a quello della foto.
Daniel-san, accorgendosi che potando una foglia da un bonsai se ne ottiene uno pi� piccolo, si domanda se sia sempre possibile ottenere dal bonsai del Maestro il risultato rappresentato nella foto.
Dato che Daniel-san non pu\`o tornare a casa prima aver finito di dare la cera alla macchina, non vuole investire troppo tempo nel capire se il compito dato dal Maestro Miyagi sia possibile o meno.
Riuscirai ad aiutare il giovane Daniel-san a raggiungere la pace interiore decidendo se il bonsai raffigurato nella foto sia riproducibile potando il bonsai del Maestro Miyagi?\\
\smallskip
\begin{flushright}
{\small{\textbf{``Nella vita bisogna fare tre cose: mettere la cera, togliere la cera, potare un bonsai.''}}\\
\textit{(Detto zen del Maestro Miyagi)}}
\end{flushright}
\bigskip

Im termini di grafi, un bonsai \`e un albero. Avr� $N$ nodi numerati da $0$ a $N-1$ ed $N-1$ archi sufficienti a garantirne la connessione.

\sezionetesto{Input}


\sezionetesto{Output}


   
% Esempi
\sezionetesto{Esempio di input/output}
\esempio{}{}

% Assunzioni
\sezionetesto{Assunzioni}
\begin{itemize}[nolistsep, noitemsep]
  \item da scrivere
\end{itemize}

\newpage
  
  \section*{Subtask}
  \begin{itemize}
    \item \textbf{Subtask 0 [0 punti]:} l'esempio del testo.
    \item \textbf{Subtask 1 [10 punti]:} da decidere
    \item \textbf{Subtask 2 [20 punti]:} da decidere
    \item \textbf{Subtask 3 [25 punti]:} da decidere
    \item \textbf{Subtask 4 [25 punti]:} da decidere
    \item \textbf{Subtask 5 [10 punti]:} da decidere
    \item \textbf{Subtask 6 [10 punti]:} nessuna restrizione (oltre quella espressa nella sezione di ``Assunzioni'' generali).
  \end{itemize}


  \section*{Note generali sul sistema di sottoposizione (con valutazione a feedback immediato) delle vostre soluzioni}

Potete sottomettere quante volte volete, vale il punteggio dell'ultima sottoposizione.
Al sistema di sottoposizione va sottomesso solo il file sorgente del vostro programma. Il nostro server compiler\`a tale sorgente avvalendosi del compilatore
suggerito dall'estensione del file da voi sottomesso.\\
\smallskip

\begin{tabular}{|l|l|l|}
\hline
  linguaggio adottato  & estensione file sottomesso & compilatore utilizzato dal server  \\ \hline
\hline               
  c++     & .cpp & g++ \\ \hline
  C       & .c   & gcc \\ \hline
\hline               
\end{tabular}\\

\smallskip
\noindent Consigliamo di testare la soluzione in locale prima di sottometterla.
Se riscontrate difformit\`a di comportamento tra quanto in locale a quanto sul server, le esatte opzioni di compilazione utilizzate dal server sul singolo problema sono riportate nella pagina del problema sul sito.

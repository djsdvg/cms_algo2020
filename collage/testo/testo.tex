\documentclass[a4paper,11pt]{article}

\usepackage[utf8x]{inputenc}
\SetUnicodeOption{mathletters}
\SetUnicodeOption{autogenerated}

\usepackage[italian]{babel}
\usepackage{booktabs}
\usepackage{mathpazo}
\usepackage{graphicx}
\usepackage[left=2cm, right=2cm, bottom=3cm]{geometry}
\frenchspacing

\begin{document}
\noindent {\Huge \textbf Arcobaleno e Collage (\texttt{collage})}\\
{\large \textbf (preso ed adattato dalla Finale Nazionale 2004 delle OII)}


\section*{Descrizione del problema}
  Nel pianeta Wobniar ogni mattina splende un bellissimo e caratteristico
arcobaleno. La particolarità consiste nella disposizione dei colori, che
possono presentarsi più volte all'interno dell'arco in una sequenza sempre nuova e sorprendente.
Ogni giorno all'alba il famoso artista Ed Esor
cattura lo splendore del nuovo arco in un collage di strisce colorate.
Per risparmiare sui materiali e meglio consentirne il riciclo,
Ed Esor cerca sempre di minimizzare
il numero di fogli di carta colorata da sovrapporre nella composizione
del collage, senza mai rinunciare a riprodurre fedelmente la sequenza
apparsa in cielo.
Aiuta Ed Esor a minimizzare il numero di fogli impiegati nel suo collage!
Se, ad esempio, l'arcobaleno fosse composto da~$3$ strisce di~$2$
colori diversi alternati, Ed Esor riuscirebbe a fare un collage usando
due soli fogli di carta: uno, disposta come base, dello stesso
colore delle due strisce alle estremità dell'arcobaleno, l'altro
posato sul centro del primo.

\section*{Dati di input}
  Il file \texttt{input.txt} ha due righe.
  La prima riga contiene solo un numero naturale $N$
  che specifica il numero di strisce dell'arcobaleno.
  La seconda riga riporta $N$ numeri interi $C_{1}$, $C_{2}$, ..., $C_{N}$
separati da spazio che specificano la sequenza di colori dell'arobaleno odierno.
Ogni colore $C_{i}$ è un numero
intero compreso tra~$0$ e~$255$.
Ampie strisce uniformi nel colore sono indicate da più numeri uguali disposti consecutivamente.

\section*{Dati di output}
  Il file \texttt{output.txt} dovrà contenere un
unico numero: il numero minimo di strisce per riprodurre l'arcobaleno.

\section*{Esempi di input/output}

  
    \noindent
    \begin{tabular}{p{11cm}|p{5cm}}
    \toprule
    \textbf{File \texttt{input.txt}}
    & \textbf{File \texttt{output.txt}}
    \\
    \midrule
    \scriptsize
    \begin{verbatim}
3 
1 2 1
\end{verbatim}
    &
    \scriptsize
    \begin{verbatim}
2
\end{verbatim}
    \\
    \bottomrule
    \end{tabular}
  
    \noindent
    \begin{tabular}{p{11cm}|p{5cm}}
    \toprule
    \textbf{File \texttt{input.txt}}
    & \textbf{File \texttt{output.txt}}
    \\
    \midrule
    \scriptsize
    \begin{verbatim}
7
1 1 2 3 1 2 1
\end{verbatim}
    &
    \scriptsize
    \begin{verbatim}
4
\end{verbatim}
    \\
    \bottomrule
    \end{tabular}

    
\section*{Assunzioni}
  \begin{itemize}
    \item al più $50$ fogli di carta sono sempre sufficienti a comporre il collage
    \item $0 < N ≤ 1\,000\,000$
    \item $0 \leq C_{i} \leq 255$ per ogni $i=1,\ldots, N$
  \end{itemize}
  
\section*{Subtask}
\begin{itemize}
\item \textbf{Subtask 1 [0 punti]:} caso di esempio.
\item \textbf{Subtask 2 [10 punti]:} $N \leq 7$, i numeri adiacenti sono sempre diversi.
\item \textbf{Subtask 3 [10 punti]:} $N \leq 7$.
\item \textbf{Subtask 4 [10 punti]:} $N \leq 12$.
\item \textbf{Subtask 5 [10 punti]:} $N \leq 20$.
\item \textbf{Subtask 6 [10 punti]:} due soli colori, $N \leq 100$, $0 \leq C_{i} \leq 1$ per ogni $i=1,\ldots, N$.
\item \textbf{Subtask 7 [10 punti]:} tre soli colori, $N \leq 100$, $0 \leq C_{i} \leq 2$ per ogni $i=1,\ldots, N$.
\item \textbf{Subtask 8 [30 punti]:} $N \leq 100$.
\item \textbf{Subtask 9 [10 punti]:} $N \le 1\,000\,000$, ma è garantita esistere una soluzione in al più $50$ fogli.  
\end{itemize}
  

\end{document}

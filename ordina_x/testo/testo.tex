% Template per generare

\documentclass[a4paper,11pt]{article}
\usepackage{lmodern}
\renewcommand*\familydefault{\sfdefault}
\usepackage{sfmath}
\usepackage[utf8]{inputenc}
\usepackage[T1]{fontenc}
\usepackage[italian]{babel}
\usepackage{indentfirst}
\usepackage{graphicx}
\usepackage{tikz}
\newcommand*\circled[1]{\tikz[baseline=(char.base)]{
    \node[shape=circle,draw,inner sep=2pt] (char) {#1};}}
\usepackage{enumitem}
% \usepackage[group-separator={\,}]{siunitx}
\usepackage[left=2cm, right=2cm, bottom=3cm]{geometry}
\frenchspacing

\newcommand{\num}[1]{#1}

% Macro varie...
\newcommand{\file}[1]{\texttt{#1}}
\renewcommand{\arraystretch}{1.3}
\newcommand{\esempio}[2]{
  \noindent\begin{minipage}{\textwidth}
    \begin{tabular}{|p{11cm}|p{5cm}|}
      \hline
      \textbf{File \file{input.txt}} & \textbf{File \file{output.txt}}\\
      \hline
      \tt \small #1 &
      \tt \small #2 \\
      \hline
    \end{tabular}
  \end{minipage}
}

% Dati del task
\newcommand{\gara}{.}
\newcommand{\nome}{Ordinare in Versione Interattiva}
\newcommand{\nomebreve}{ordina\_x}

\begin{document}
  
  
  % Intestazione
  \noindent{\Large \gara}
  \vspace{0.5cm}
  
  \noindent{\Huge \textbf \nome~(\texttt{\nomebreve})}
  \vspace{0.2cm}\\
  
  % Descrizione del task
  \section*{Descrizione del problema}
    
  \noindent
  Ti viene fornito un set di $n$ biglie numerate da $0$ ad $n-1$,
  tutte di peso diverso.
  Devi scrivere una procedura che riordini le biglie per peso crescente.
  Il file sorgente da sottomettere deve avere la seguente struttura:
\begin{verbatim}
#include "ourLibToPlay.h"

void ordina(long int n, long int label[]) {
   ...
} 
\end{verbatim}

  \noindent
  Il parametro $n$ che viene passato alla funzione \texttt{ordina} da te realizzata \`e il numero di biglie sotto esame. Le $n$ biglie differiscono tutte per peso.
  Per riordinarle, potrai servirti di una bilancia a braccia eguali
  invocando, dalla tua implementazione della procedura \texttt{ordina},
  la seguente funzione:
  
  \vspace{0.2cm}
  
  \noindent
  \texttt{int pesa(long int bigliaA, long int bigliaB)}
  
  \vspace{0.2cm}
  
  \noindent
  La funzione \texttt{pesa} ritorna
  \begin{itemize}[noitemsep]
    \item $-1$, se \texttt{bigliaA} \`e pi\`u leggera di \texttt{bigliaB};
    \item $1$,  se \texttt{bigliaA} \`e pi\`u pesante di \texttt{bigliaB}.
  \end{itemize}

  \noindent
  Una volta individuato il corretto ordinamento delle biglie,
  devi consegnarlo invocando la funzione:
  
  \vspace{0.2cm}
  
  \noindent
  \texttt{void consegnaBiglieInOrdine( long int biglia\_in\_pos[] )}
  
  
  \section*{Subtask}
  \begin{itemize}
    \item \textbf{Subtask 0 [2 punti]:} le biglie sono ordinate per peso crescente da $0$ a $n-1$, $n \leq 1000$.
    \item \textbf{Subtask 1 [3 punti]:} le biglie sono ordinate per peso crescente oppure per peso decrescente, $n \leq 1000$.
    \item \textbf{Subtask 2 [25 punti]:} sono consentite al pi\`u $n(n-1)/2$ pesate, $n \leq 1000$.
    \item \textbf{Subtask 3 [30 punti]:} sono consentite al pi\`u $5n\log_2 n$ pesate, $n \leq 1000$.
    \item \textbf{Subtask 4 [20 punti]:} sono consentite al pi\`u $n\log_2 n$ pesate, $n \leq 1000$.
    \item \textbf{Subtask 5 [20 punti]:} sono consentite al pi\`u $n(\lceil \log_2 n \rceil -1)$ pesate e devi ordinare in al pi\`u 1 secondo fino a $100,000$ di biglie.
  \end{itemize}
  
  % Assunzioni
  \section*{Assunzioni}
  \begin{itemize}[nolistsep, noitemsep]
    \item Il programma termina dopo la prima chiamata alla funzione \texttt{consegnaBiglieInOrdine} oppure allo scadere del tempo limite.
    \item $1 \le n \le 100\,000$.
  \end{itemize}


  % Cosa sottomettere
  \section*{Cosa deve contenere il File da Sottomettere, e come viene gestito dal server}

  L'estensione del file che sottometti, *.c, *.cpp oppure *.pas,
  determina se esso viene compilato col compilatore del C (gcc),
  del c++ (g++) oppure del PASCAL (fpc).
  Le opzioni di compilazione, specifiche al linguaggio,
  sono riportate nel dettaglio sulla pagina "Descrizione" del problema
  in modo che tu possa simulare esattamente in locale cosa succede sul server.
  I comandi di compilazione riportati assumono che il nome del file sottomesso (privato di estensione) coincida col nome del problema.

  Nel caso di un problema interattivo come questo,
  dove il tuo programma interagisce con altri da noi predisposti
  (il grader ed anche un header da includere), i comandi di compilazione riportati compilano ed assemblano i vari pezzi (il tuo programma, il grader, e l'header).
  Se vuoi testare il comportamento del tutto in locale,
  devi allora scaricarti il nostro grader e l'header che rendiamo disponibile tra gli allegati alla pagina del problema.

  Tra gli allegati potremmo inoltre fornire una soluzione di esempio che,
  pur essendo compilabile sia in locale che dal server (ove ad esso sottomessa),
  faccia pochi o nessun punto in quanto non significativa per il problema
  in questione.
  
\end{document}

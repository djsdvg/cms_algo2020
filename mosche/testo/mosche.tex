\renewcommand{\nomebreve}{mosche}
\renewcommand{\titolo}{Mosche (da coci 2015-16, round 7, task 2)}

\introduzione{}

Gli $R\times S$ pixel di un monitor sono organizzati su $R$ righe ed $S$ colonne.
Abbiamo lasciata aperta la finestra, e sono entrate delle mosche che sono andate a posizionarsi ipnotizzate su alcuni di questi pixel.
Una mosca per pixel, ma alcuni pixel sono rimasti scoperti.
Abbiamo una racchetta di forma quadrata, tipo $K\times K$.
Vogliamo uccidere il maggior numero di mosche con un solo colpo di racchetta,
dove piazzare il colpo?
Nota che la racchetta $K\times K$ deve ricadere interamente entro il monitor $R\times S$,
poich\`e il bordo del monitor sporge rialzato.
Inoltre, le mosche posizionate in corrispondenza del bordo della racchetta
fanno in tempo a sfuggire al colpo. 


\sezionetesto{Dati di input}
La prima riga in input contiene, nell'ordine,
i tre numeri interi $R$, $S$ e $K$.
Le successive $R$ righe offrono una descrizione
della disposizione delle mosche sul monitor.
Denotiamo col carattere '\,.\,' un pixel libero, mentre la presenza di una mosca
\`e denunciata dal carattere '*' nella sua posizione sul monitor.

\sezionetesto{Dati di output}
La prima riga in output deve contenere un intero
che rappresenta il massimo numero di mosche abbattibili in un sol colpo.
Se questa riga \`e corretta l'istanza viene considerata risolta almeno per met\`a.
Per completare la soluzione,
si rappresenti nelle successive $R$ righe il collocamento della racchetta
sul monitor, come mostrato negli esempi. 


% Esempi
\sezionetesto{Esempio di input/output}
\esempio{
3 5 3

.....

.*.*.

.....
}{
1

+-+..

|*|*.

+-+..
}

\esempio{
7 6 4

......

.*.*.*

......

.*.*..

..*...

..*...

*....*
}{
2

......

.*.*.*

+{-}-+..

|*.|..

|.*|..

+{-}-+..

*....*
}

\esempio{
9 9 6

***......

......*.*

.*....*..

..*...*..

..*.*....

..*....*.

.....*...

.*...***.

.........
}{
6

***......

......*.*

.*....*..

..*+{-}{-}{-}{-}+

..*|*...|

..*|...*|

...|.*..|

.*.|.***|

...+{-}{-}{-}{-}+
}

% Assunzioni
\sezionetesto{Assunzioni e note}
\begin{itemize}[nolistsep, noitemsep]
  \item $3 \le R,S,K \le 2\,000$.
\end{itemize}

  \section*{Subtasks}
  \begin{itemize}
    \item \textbf{Subtask 0 [1 punto]:} i tre esempi del testo.
    \item \textbf{Subtask 1 [29 punti]:} gli altri 20 esempi alle COCI.
    \item \textbf{Subtask 2 [30 punti]:} $K \le 100$, $R,S  \le 2\,000$.
    \item \textbf{Subtask 3 [40 punti]:} $3 \le K \le R,S  \le 2\,000$.
  \end{itemize}
  

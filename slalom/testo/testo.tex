
\documentclass[a4paper,11pt]{article}

\usepackage[utf8x]{inputenc}
\SetUnicodeOption{mathletters}
\SetUnicodeOption{autogenerated}

\usepackage[italian]{babel}
\usepackage{booktabs}
\usepackage{mathpazo}
\usepackage{graphicx}
\usepackage[left=2cm, right=2cm, bottom=3cm]{geometry}
\frenchspacing

\begin{document}
\noindent {\Large Selezioni nazionali 2006}
\vspace{0.5cm}

\noindent {\Huge Slalom tra le telecamere (\texttt{slalom})}


\vspace{0.5cm}
\noindent {\Large Difficoltà D = 2 (tempo limite 3 sec).}

\section*{Descrizione del problema}
   
Alle Olimpiadi invernali è stata creata una struttura
comprendente varie piste da fondo, tutte percorribili in entrambe le
direzioni e raccordate tramite aree di snodo.  Da ogni snodo è
possibile imboccare una o più piste, mentre ogni pista collega
due snodi distinti disposti ai suoi due estremi (e la pista non
contiene altri snodi al suo interno). La struttura
permette agli atleti di spostarsi da ogni snodo a ogni altro snodo
muovendosi sempre su pista (ovvero, senza togliersi gli sci dai
piedi).  Inoltre, il percorso da seguire per passare da un certo snodo
a un qualunque altro snodo è unico per ogni coppia di snodi (in
pratica, la struttura non contiene cicli).

I dirigenti della televisione devono valutare dove localizzare le
postazioni fisse delle telecamere, da piazzare in alcuni degli snodi
della struttura, in modo da poter riprendere tutte le gare.  Gli snodi
vengono numerati da 1 a $N$, e la pista avente gli snodi
$i$ e $j$ come estremi viene indicata tramite la
coppia di interi $i$ e $j$.

Nel valutare il costo delle postazioni fisse per le telecamere, i
dirigenti hanno stimato il costo di installazione per ciascuno snodo
(tali costi non sono necessariamente uguali e possono variare da snodo
a snodo). Inoltre, per ogni pista, \emph{almeno uno} tra i due snodi
che essa collega deve contenere una telecamera per garantire la
copertura televisiva.
Aiutate i dirigenti a identificare gli snodi che danno luogo alla
copertura televisiva di costo minimo, calcolato come la somma dei
costi di installazione delle telecamere in tali snodi.


\section*{Dati di input}
  La prima riga del file \texttt{input.txt} contiene l'intero
positivo $N$, il numero di snodi nella struttura.La riga successiva contiene $N$ interi positivi separati da uno
spazio per rappresentare, nell'ordine, i costi per l'installazione delle
telecamere negli snodi 1,2,...,$N$.Le successive $N - 1$ righe contengono coppie di interi
positivi che rappresentano le piste. Ogni riga è composta da
due interi distinti $i$ e $j$ separati da uno
spazio, a rappresentare la pista avente gli snodi $i$ e
$j$ come estremi. Da notare che la stessa pista non
può apparire in più di una riga e l'ordine dei due
interi specificati per la pista non è significativo (sia la
coppia $i$ e $j$ che la coppia $j$ e
$i$ rappresentano la medesima pista).

\section*{Dati di output}
  Il file \texttt{output.txt} è composto da due righe.
La prima riga contiene un intero positivo che rappresenta il numero
$P$ di snodi che garantiscono la copertura televisiva con il
minimo costo (ossia, tali snodi minimizzano la somma dei costi di
installazione delle telecamere).
La seconda riga contiene $P$
interi positivi separati da uno spazio per indicare quali sono gli snodi
coinvolti in tale copertura.

Da notare che possono esistere più soluzioni valide e, in tal
caso, è sufficiente restituirne una qualunque come risposta
(come mostrato negli esempi).

  \section*{Assunzioni}
  \begin{itemize}
  
    \item  2 ≤ $N$ ≤ 400000
    \item  1 ≤ $P$ ≤ $N$
  \end{itemize}

\section*{Esempi di input/output}

  
    \noindent
    \begin{tabular}{p{11cm}|p{5cm}}
    \toprule
    \textbf{File \texttt{input.txt}}
    & \textbf{File \texttt{output.txt}}
    \\
    \midrule
    \scriptsize
    \begin{verbatim}
5
1 2 1 1 1
4 2
1 2
3 2
5 4
\end{verbatim}
    &
    \scriptsize
    \begin{verbatim}
3
1 4 3 
\end{verbatim}
    \\
    \bottomrule
    \end{tabular}
  
    \noindent
    \begin{tabular}{p{11cm}|p{5cm}}
    \toprule
    \textbf{File \texttt{input.txt}}
    & \textbf{File \texttt{output.txt}}
    \\
    \midrule
    \scriptsize
    \begin{verbatim}
5
1 2 1 1 1
4 2
1 2
3 2
5 4
\end{verbatim}
    &
    \scriptsize
    \begin{verbatim}
2
2 5 
\end{verbatim}
    \\
    \bottomrule
    \end{tabular}
  
\section*{Nota/e}
\begin{itemize}
  
    \item 
Se usate la piattaforma di sviluppo software basata sul compilatore
Turbo Pascal e sul sistema operativo Windows, fare attenzione: i
vostri programmi potrebbero essere valutati in una piattaforma diversa
dalla vostra, e la garanzia di uniformità di comportamenti si
ha soltanto se utilizzate \emph{sempre} il tipo \texttt{LongInt}
al posto del tipo \texttt{Integer} (quest'ultimo permette di
rappresentare gli interi nell'intervallo [-32768...32767] mentre
\texttt{LongInt} ne permette la rappresentazione in
[-2147483648...2147483647]).

\end{itemize}



\end{document}

% Template per generare 

\documentclass[a4paper,11pt]{article}
\usepackage{lmodern}
\renewcommand*\familydefault{\sfdefault}
\usepackage[utf8]{inputenc}
\usepackage[T1]{fontenc}
\usepackage[italian]{babel}
\usepackage{indentfirst}
\usepackage{graphicx}
\usepackage{tikz}
\newcommand*\circled[1]{\tikz[baseline=(char.base)]{
		\node[shape=circle,draw,inner sep=2pt] (char) {#1};}}
% \usepackage[group-separator={\,}]{siunitx}
\usepackage[left=2cm, right=2cm, bottom=3cm]{geometry}
\frenchspacing

\newcommand{\num}[1]{#1}

% Macro varie...
\newcommand{\file}[1]{\texttt{#1}}
\renewcommand{\arraystretch}{1.3}
\newcommand{\esempio}[2]{
\noindent\begin{minipage}{\textwidth}
\begin{tabular}{|p{11cm}|p{5cm}|}
	\hline
	\textbf{File \file{input.txt}} & \textbf{File \file{output.txt}}\\
	\hline
	\tt \small #1 &
	\tt \small #2 \\
	\hline
\end{tabular}
\end{minipage}
}

% Dati del task
\newcommand{\gara}{algo2015}
\newcommand{\nome}{Piastrellature di un corridoio $2\times n$ }
\newcommand{\nomebreve}{piastrelle2}

\begin{document}
% Intestazione
\noindent{\Large \gara}
\vspace{0.5cm}

\noindent{\Huge \textbf \nome~(\texttt{\nomebreve})}

% Descrizione del task
\section*{Descrizione del problema}
Abbiamo un corridoio di dimensione $2 \times N$ da piastellare.
Sono a disposizione solamente piastrelle di dimensioni $1 \times 1$ e $1 \times 2$; queste ultime possono essere ruotate di 90 gradi. Vorremmo conoscere tutte le possibili disposizioni delle piastrelle sul pavimento del corridoio $2 \times N$.
Lista le possibili piastrellature una alla volta.
Dato che ogni piastrella sar\`a adiacente ad al pi\`u $4$ piastrelle,
risulta possibile labellare le piastrelle con i colori $\{1,2,3,4,5\}$
in modo che piastrelle adiacenti abbiano colore diverso.
Per specificare una singola piastrellatura del corridoio (tiling della griglia $2 \times N$ con pezzi di domino, ossia con sottogriglie $1\times 2$ oppure $2\times 1$), specifica solo il colore per ciascuna delle $2N$ celle del corridoio. 

% Input
\section*{File di input}
Il programma deve leggere da un file di nome \file{input.txt} un unico intero $N$, che determina la dimensione del corridoio.

% Output
\section*{File di output}
Il programma deve scrivere in un file di nome \file{output.txt}.
La prima riga del file contiene un intero $K$, il numero di possibili piastrellature.
Il file va organizzato in $3K$ righe.
Per ogni $i=1,\ldots,K$,
le righe $3i-1$ e $3i$ rappresentano l'$i$-esima piastrellatura,
e la riga successiva a queste serve come spaziatura.
Per specificare l'$i$-esima piastrellatura,
si prenda un qualsiasi $5$-coloring della stessa e lo si riporti
sulle due righe nel modo ovvio (rispettando cio\`e la topologia sottostante).

% Assunzioni
\section*{Assunzioni}

\begin{itemize}
\item tempo massimo: $2$ secondi.
\end{itemize}

% Subtasks
\section*{Subtask}
\begin{itemize}
\item \textbf{Subtask 1 [5 punti]:} caso di esempio.
\item \textbf{Subtask 2 [15 punti]:} $N \le 5$.
\item \textbf{Subtask 3 [30 punti]:} $N \le 8$.
\item \textbf{Subtask 4 [40 punti]:} $N \le 9$.
\item \textbf{Subtask 5 [10 punti]:} $N \le 10$.
\end{itemize}

% Esempi
\section*{Esempio di input/output}
\esempio{3}{\scriptsize
22

1 2 1

2 1 2

\ 

1 2 3 

2 1 3 

\  

1 2 3 

2 1 1 

\  

1 2 2 

2 1 3 

\  

1 2 2 

2 1 1 

\  

1 3 1 

2 3 2 

\  

1 3 1 

2 3 1 

\  

1 3 1 

2 2 3 

\  

1 3 1 

2 2 1 

\  

1 3 3 

2 2 1 

\  

1 1 2 

2 3 1 

\  

1 1 2 

2 3 2 

\  

1 1 2 

2 3 3 

\  

1 1 2 

2 2 1 

\  

1 1 3 

2 2 3 

\  

1 2 1 

1 3 2 

\  

1 2 1 

1 3 1 

\  

1 2 1 

1 3 3 

\  

1 2 2 

1 3 1 

\  

1 2 2 

1 3 3 

\  

1 2 1

1 2 3 

\ 

1 2 1

1 2 1 

}

\end{document}


\documentclass[a4paper,11pt]{article}

\usepackage[utf8x]{inputenc}
\SetUnicodeOption{mathletters}
\SetUnicodeOption{autogenerated}

\usepackage[italian]{babel}
\usepackage{booktabs}
\usepackage{mathpazo}
\usepackage{graphicx}
\usepackage[left=2cm, right=2cm, bottom=3cm]{geometry}
\frenchspacing

\begin{document}
\noindent {\Large Selezioni nazionali 2007}
\vspace{0.5cm}

\noindent {\Huge Somme di sequenze (\texttt{somme})}


\vspace{0.5cm}
\noindent {\Large Difficoltà D = 3 (tempo limite 3 sec).}

\section*{Descrizione del problema}
  
Data una sequenza $S$ di $N$ numeri interi, tipo 
\texttt{11 -4 52 -7 -2 -20}, 
vogliamo computare la somma di tutti i numeri in $S$ avvalendoci di un
robot con capacità limitate. Infatti, tale robot può
soltanto effettuare la somma intermedia $Y$ di due numeri $A$ e $B$
consecutivi nella sequenza, rimpiazzando $A$ e $B$ con $Y$ e ottenendo
così una sequenza più corta (con un intero in meno).

Per effettuare tale somma intermedia $Y$ e produrre la sequenza
più corta, il robot consuma esattamente $|Y|$ unità di
energia (dove $|Y|$ indica il valore assoluto di un numero intero $Y$).

Per calcolare la somma degli $N$ numeri in $S$ sono quindi necessarie $N-1$
somme intermedie. Pur disponendo di energia sufficiente per eseguire
le $N-1$ somme intermedie in tale calcolo, il robot ha problemi con i
picchi di energia. In altre parole, vogliamo che il massimo consumo
energetico per una somma intermedia (il picco energetico) sia
minimizzato. 

Nel caso di cui sopra una soluzione ottima è data da
\texttt{11   -4  52   -7 -22} $\rightarrow$
\texttt{11   -4  52 -29} $\rightarrow$
\texttt{11   -4  23} $\rightarrow$
\texttt{7 23} $\rightarrow$ \texttt{30}
In questo caso i valori intermedi ottenuti sono -22, -29, 23, 7, 30 e
il picco energetico è $30$, essendo il massimo tra $|-22|$, $|-29|$,
$|23|$, $|7|$ e $|30|$. Meglio di così non si può fare in
quanto il valore della somma è $30$.
Si analizzino gli ulteriori esempi forniti sotto.

Scrivete un programma che calcoli il minimo picco energetico per una
sequenza di interi.


\section*{Dati di input}
  
Il file \texttt{input.txt} è composto da due righe.

La prima riga contiene un intero positivo che rappresenta il numero N
di interi nella sequenza d'ingresso.

La successiva riga contiene N interi, separati da uno spazio, che
rappresentano la sequenza su cui computare la somma.


\section*{Dati di output}
  Il file \texttt{output.txt} è composto da una sola riga
che contiene l'intero non negativo E, il quale rappresenta il picco
energetico minimo del robot per calcolare la somma degli interi nella
sequenza d'ingresso.

  \section*{Assunzioni}
  \begin{itemize}
  
    \item $ 2 ≤ N ≤ 500 $
  \end{itemize}

\section*{Esempi di input/output}

  
    \noindent
    \begin{tabular}{p{11cm}|p{5cm}}
    \toprule
    \textbf{File \texttt{input.txt}}
    & \textbf{File \texttt{output.txt}}
    \\
    \midrule
    \scriptsize
    \begin{verbatim}
6
11 -4 52 -7 -2 -20
\end{verbatim}
    &
    \scriptsize
    \begin{verbatim}
30
\end{verbatim}
    \\
    \bottomrule
    \end{tabular}
  
    \noindent
    \begin{tabular}{p{11cm}|p{5cm}}
    \toprule
    \textbf{File \texttt{input.txt}}
    & \textbf{File \texttt{output.txt}}
    \\
    \midrule
    \scriptsize
    \begin{verbatim}
5
4 7 -9 8 -10
\end{verbatim}
    &
    \scriptsize
    \begin{verbatim}
2
\end{verbatim}
    \\
    \bottomrule
    \end{tabular}
  
    \noindent
    \begin{tabular}{p{11cm}|p{5cm}}
    \toprule
    \textbf{File \texttt{input.txt}}
    & \textbf{File \texttt{output.txt}}
    \\
    \midrule
    \scriptsize
    \begin{verbatim}
3
7 -1 -8
\end{verbatim}
    &
    \scriptsize
    \begin{verbatim}
6
\end{verbatim}
    \\
    \bottomrule
    \end{tabular}
  
    \noindent
    \begin{tabular}{p{11cm}|p{5cm}}
    \toprule
    \textbf{File \texttt{input.txt}}
    & \textbf{File \texttt{output.txt}}
    \\
    \midrule
    \scriptsize
    \begin{verbatim}
5
0 0 0 0 0
\end{verbatim}
    &
    \scriptsize
    \begin{verbatim}
0
\end{verbatim}
    \\
    \bottomrule
    \end{tabular}
  
\section*{Nota/e}
\begin{itemize}
  
    \item 
Se usate la piattaforma di sviluppo software basata sul compilatore
Turbo Pascal e sul sistema operativo Windows, fare attenzione: i
vostri programmi potrebbero essere valutati in una piattaforma diversa
dalla vostra, e la garanzia di uniformità di comportamenti si
ha soltanto se utilizzate \emph{sempre} il tipo \texttt{LongInt}
al posto del tipo \texttt{Integer} (quest'ultimo permette di
rappresentare gli interi nell'intervallo [-32768...32767] mentre
\texttt{LongInt} ne permette la rappresentazione in
[-2147483648...2147483647]).

\end{itemize}



\end{document}

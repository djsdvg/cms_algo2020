
\documentclass[a4paper,11pt]{article}

\usepackage[utf8x]{inputenc}
\SetUnicodeOption{mathletters}
\SetUnicodeOption{autogenerated}

\usepackage[italian]{babel}
\usepackage{booktabs}
\usepackage{mathpazo}
\usepackage{graphicx}
\usepackage[left=2cm, right=2cm, bottom=3cm]{geometry}
\frenchspacing

\begin{document}

\noindent {\Huge longwalk (\texttt{longwalk})}


\vspace{0.5cm}
\noindent {\Large (tempo limite 1 sec)}

\section*{Descrizione del problema}

    Abbiamo un grafo con $N$ nodi, numerati da 1 a $N$, e $M$ archi diretti.\\
    Vogliamo trovare una di queste due cose nel grafo:
    \begin{itemize}
        \item Un ciclo di qualsiasi dimensione
        \item Se non esistono cicli, il cammino più lungo che è possibile fare nel grafo
    \end{itemize}

\section*{Dati di input}

      Il file di input è composto da $M+1$ righe.

      La prima riga contiene due numeri interi $N$ e $M$,
      rispettivamente il numero di nodi ed il numero di archi

      Ciascuna delle $M$ righe successive contiene due numeri interi
      $p$ e $q$ e rappresenta un arco diretto tra il nodo $p$ e il nodo $q$.

\section*{Dati di output}

      Il file di output è composto da 2 righe.

      Qualora esista una ciclo di $k$ nodi $n_{1}, \ldots, n_{k}$,
      la prima riga deve contenere gli interi $-1$ e $k$
      mentre la seconda riga $k$ numeri interi $n_{1}, \ldots, n_{k}$
      che rappresentano gli indici dei nodi nell'ordine di percorrenza del ciclo.

      Altrimenti, la prima riga deve contenere
      la lunghezza $\overline{k}$ del cammino massimo che è possibile fare nel grafo e
      la seconda riga $\overline{k}$ numeri interi $n_{1}, \ldots, n_{\overline{k}}$
      che rappresentino una tale sequenza.

  \section*{Assunzioni}
  \begin{itemize}

    \item $2 ≤ N ≤  100000$
    \item $2 ≤ M ≤ 1000000$
  \end{itemize}

\section*{Esempi di input/output}


    \noindent
    \begin{tabular}{p{11cm}|p{5cm}}
    \toprule
    \textbf{File \texttt{input.txt}}
    & \textbf{File \texttt{output.txt}}
    \\
    \midrule
    \scriptsize
    \begin{verbatim}
5 9
1 4
4 2
5 4
1 4
1 4
4 2
4 2
5 4
1 5
      \end{verbatim}
    &
    \scriptsize
    \begin{verbatim}
4
1 5 4 2
      \end{verbatim}
    \\
    \bottomrule
    \end{tabular}

    \noindent
    \begin{tabular}{p{11cm}|p{5cm}}
    \toprule
    \textbf{File \texttt{input.txt}}
    & \textbf{File \texttt{output.txt}}
    \\
    \midrule
    \scriptsize
    \begin{verbatim}
5 11
5 4
5 3
2 3
4 3
2 5
1 2
4 2
1 2
4 3
1 2
2 3
      \end{verbatim}
    &
    \scriptsize
    \begin{verbatim}
−1 3
2 5 4
      \end{verbatim}
    \\
    \bottomrule
    \end{tabular}



\end{document}
